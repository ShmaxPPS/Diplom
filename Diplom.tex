\documentclass[a4paper]{article}
\usepackage[14pt]{extsizes}
\usepackage[utf8]{inputenc}
\usepackage[T2A]{fontenc}
\usepackage[russian]{babel}
\usepackage{setspace,amsmath}
\usepackage{amssymb, amsthm}
\usepackage{mathtext}
\usepackage{mathenv}
\usepackage{tocloft}
\usepackage{enumitem}
\usepackage[indentfirst]{titlesec}
\usepackage[left=30mm, top=20mm, right=20mm, bottom=20mm, nohead, footskip=10mm]{geometry}

\onehalfspacing

\makeatletter
\renewcommand{\cftsecleader}{\cftdotfill{\cftdotsep}}

\newtheorem*{mproblem}{Условие}
\newtheorem*{mdefinition}{Определение}
\newtheorem*{mlemma}{Лемма}
\newtheorem*{mtheorem}{Теорема}
\newtheorem*{msolution}{Доказательство}

\begin{document}
\begin{titlepage}
	\begin{center}
		\hfill \break
		\large{Министерство образования и науки Российской Федерации}\\
		\hfill \break
		\normalsize{Государственное образовательное учреждение}\\ 
		\normalsize{высшего профессионального образования}\\
		\normalsize{«Московский физико-технический институт (государственный университет)»}\\
		\normalsize{Факультет инноваций и высоких технологий}\\
		\normalsize{Кафедра анализа данных}\\
		\hfill \break
		\hfill \break
		\hfill \break
		\Large{\textbf{Магистерская диссертация}}\\
		\hfill \break
		\large{Тема: \textbf{Название моей работы (TODO)}}\\
	\end{center}

	\begin{flushright}
		\hfill \break
		Направление:  010400\\
		Прикладные математика и информатика\\
		\hfill \break
		\hfill \break
		\hfill \break
	\end{flushright}
	
	\begin{flushright}
		\normalsize{
			\begin{tabular}{rcr}
				Выполнил:\ \\ студент 093 группы & \underline{\hspace{3cm}} & Попов М.В. \\\\
				Научный руководитель:\ \\ д.физ.-мат.н., проф.(todo) & \underline{\hspace{3cm}}& Ромащенко А.Е. \\\\
			\end{tabular}
		}
	\end{flushright}

	\begin{center}
		\hfill \break
	\end{center}

	\begin{center} г. Москва 2016 \end{center}
	\thispagestyle{empty}
\end{titlepage}
 
\newpage

\tableofcontents

\newpage
 
\newpage


\setcounter{section}{0}
\section*{Введение}
\addcontentsline{toc}{section}{Введение}
(ToDo) Актуальность, новизна, краткая выжимка.

\addtocounter{section}{1}
\section*{Коммуникационная сложность}
\addcontentsline{toc}{section}{Коммуникационная сложность}
\setcounter{subsection}{0}

\subsection{Постановка задачи}
Мы будем рассматривать задачи следующего вида: пусть имеется два человека, которые хотят совместно
вычислить значение некоторой функции от двух переменных $f(x, y)$. По традиции мы будем называть
первого участника игры Алисой, а второго Бобом. Сложность у этой задачи в том, что Алиса знает только
значение аргумента $x$, а Боб значение аргумента $y$. Алиса и Боб могут обмениваться сообщениями 
по каналу связи. Требуется вычислить значение $f(x, y)$, переслав по каналу связи минимальное
количество информации.

Мы предполагаем, что Алиса и Боб заранее (до того, как им станут известны значения $x$ и $y$)
договариваются о коммуникационном протоколе --- о наборе соглашений, какие именно данные и
в каком порядке они будут пересылать друг другу при тех или иных значениях $x$ и $y$.

Опишем теперь всю задачу более формально. Пусть имеются конечные множества $X, Y, Z$ и задана
некоторая функция $f:X\times Y\rightarrow Z$.

\begin{mdefinition}
    Коммуникационным протоколом для вычисления некоторой функции $f:X\times Y\rightarrow Z$ называется
    ориентированное двоичное дерево со следующей разметкой на вершинах и ребрах:
    \begin{itemize}[noitemsep]
        \item каждая нелистовая вершина помечена буквой $A$ или $B$;
        \begin{itemize}[noitemsep]
			\item у вершин с пометкой $A$ определена функция $g_i:X\rightarrow \{0,1\}$;
			\item у вершин с пометкой $B$ определена функция $f_j:Y\rightarrow \{0,1\}$;
        \end{itemize}
        \item каждой листовой вершине сопоставлен элемент множеста $Z$;
        \item каждое ребро помечено $0$ или $1$.

    \end{itemize}
\end{mdefinition}

Пусть Алиса и Боб договорились, что будут действовать по некоторому протоколу $\mathcal{P}$. Затем
Алиса получила $x\in X$, а Боб получил $y\in Y$. Поместим фишку в корневую вершину нашего протокола
$\mathcal{P}$ и будем перемещать ее вниз по дереву, последовательно удаляясь от корня,
пока она не попадём в один из листьев. Перемещение фишки выполняется следующим образом. Если текущая 
вершина помечена буквой $A$ это значит, что сейчас очередь Алисы. Она применяет функцию $g_i$ текущей 
вершины к своему значению $x$. Алиса отправляет по каналу связи бит равный $g_i(x)$ и перемещает
фишку по ребру, помеченному как $g_i(x)$. Боб получает отправленный бит и понимает куда была сдвинута фишка.
Для вершин помеченных буквой $B$ поступают аналогично. Когда фишка попадает в лист дерева,
записанное там значение $z\in Z$ объявляется результатом выполнения протокола.

Мы говорим, что протокол $\mathcal{P}$ вычисляет функцию $f:X\times Y \rightarrow Z$, если для любого
$x\in X$ и любого $y\in Y$ при движении из корня по пути, соответствующему заданным $x$ и $y$,
мы попадаем в лист, помеченный $z=f(x,y)$.

\begin{mdefinition}
    Сложностью коммуникационного протокола называется его глубина. Коммуникационной сложностью функции
$f$ называется минимальная сложность протокола, вычисляющего $f$. Мы будем обозначать её $CC(f)$.
\end{mdefinition}


\subsection{Подсекция}

\addtocounter{section}{1}
\section*{Коммуникационная сложность2}
\addcontentsline{toc}{section}{Коммуникационная сложность}
\setcounter{subsection}{0}

\subsection{Постановка задачи}
\end{document}
