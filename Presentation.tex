\documentclass[utf8]{beamer}
\usepackage [utf8]{inputenc}
\usepackage[russian]{babel}
\usepackage[T2A]{fontenc}
\usepackage{graphicx}
\usepackage{wrapfig}
\usepackage{listings}
\usepackage{color}
\usepackage{array}
\usepackage{lipsum}

\newsavebox{\lonbeamergestsec}

\usetheme{Madrid}
\useoutertheme{tree}

\newtheorem{mdefinition}{Определение}[section]
\newtheorem{mproblem}{Задача}[subsection]
\newtheorem{mremark}{Примечание}[subsection]
\newtheorem{msuggest}{Предложение}[subsection]
\newtheorem{magree}{Соглашение}[subsection]
\newtheorem{mclaim}{Утверждение}[subsection]
\newtheorem{mlemma}{Лемма}[subsection]
\newtheorem{mtheorem}{Теорема}
\newtheorem{mconseq}{Следствие}

\definecolor{dkgreen}{rgb}{0,0.6,0}
\definecolor{gray}{rgb}{0.5,0.5,0.5}
\definecolor{mauve}{rgb}{0.58,0,0.82}


\title{Сравнительный анализ методов оценивания бикликового покрытия}
\date{23 июня 2016}
\author{Попов Максим\ \\ 6 Курс ФИВТ\ \\ Кафедра: Анализ данных}


\begin{document}

    \begin{frame}
		\titlepage
	\end{frame}
	
	\section{Определения и понятия}
	\begin{frame}
		\frametitle{Определения и понятия}
		\begin{mdefinition}
		    \textbf{Бикликой} неориентированного графа называется подмножество его вершин, образующих полный 
		    двудольный подграф.
		\end{mdefinition}
		
		\begin{mdefinition}
		    \textbf{Бикликовым покрытием} $\boldsymbol{bcc(G)}$ графа $G$ будем называть наименьшее число, возможно, пересекающихся
		    биклик, которыми можно покрыть все ребра графа $G$.
		\end{mdefinition}
	\end{frame}
	
	\section{Мотивировка}
	\begin{frame}
		\frametitle{Мотивировка}
		\begin{itemize}
		    \item[1)] Естественный вопрос теории графов + приложения в вычислительных задачах.
		    \item[2)] Приложения в коммуникационной сложности:
		    \medskip
		    \begin{itemize}
				\item $bcc(G)$ дает \textbf{нижнюю} оценку для \textbf{детерминированной} коммуникационной сложности;
				\item $bcc(G)$ дает \textbf{точную} оценку для \textbf{недетерминированной} коммуникационной сложности.
		    \end{itemize}
		\end{itemize}
	\end{frame}
	
	\section{Методы оценивания}
	\begin{frame}
		\frametitle{Методы оценивания}
		Нас интересует нижняя оценка величины $bcc(G)$.
		\medskip
		
		Сравниваем 3 метода доказательства оценок:
		\begin{itemize}
		    \item[1)] \textbf{Трудного множества};
		    \item[2)] \textbf{Куликова-Юкны};
		    \item[3)] \textbf{Информационных неравенств}.
		\end{itemize}

	\end{frame}
	
	\section{Основные результаты}
	\begin{frame}
		\frametitle{Основные результаты}
		\begin{itemize}
		    \item[1)] Сравнил методы оценивания на графах, построенных при помощи геометрических конфигураций;
		    \item[2)] Доказал, что метод трудного множества для произвольного графа дает оценки не хуже 
		    метода Куликова-Юкны;
		    \item[3)] С помощью метода случайных графов показал, что техника трудного множества может 
		    давать оценки значительно сильнее, чем метод Куликова-Юкны;
		    \item[4)] Получил обобщение метода информационных неравенств для задачи коммуникационной 
		    сложности с $m > 2$ участниками.
		\end{itemize}
	\end{frame}
	
	\section{Методы оценивания}
	\begin{frame}
		\frametitle{Метод трудного множества}
		\begin{mdefinition}
		     В графе $G$ подмножество рёбер $S$ называем \textbf{трудным}, если для любых двух рёбер $(x,y)\in S$ и $(x',y')\in S$ 
		     \begin{itemize}
		         \item либо $(x,y')$ не является ребром,
		         \item либо $(x',y)$ не является ребром.
		     \end{itemize}

			$\boldsymbol{fool(G)}$ -- размер максимального трудного множества.
		\end{mdefinition}
		
		\begin{mtheorem}
		    Если $S$ трудное множество графа $G$, то $bcc(G) \geq |S|$. В частности, $$bcc(G) \geq fool(G).$$  
		\end{mtheorem}
	\end{frame}
	
	\begin{frame}
		\frametitle{Метод Куликова-Юкны}
		Jukna S., Kulikov A. S. On covering graphs by complete bipartite subgraphs. 2009.
		
		\begin{mtheorem}
		    Для произвольного графа $G = (V, E)$ $$bcc(G) \geq \frac{\nu(G)^2}{|E|},$$ 
		    где $\nu(G)$ -- размер максимального паросочетания в $G$.
		\end{mtheorem}
	\end{frame}
	
	\begin{frame}
		\frametitle{Метод информационных неравенств}
		Kaced T., Romashchenko A.E., Vereshchagin N.K. Conditional Information Inequalities and Combinatorial Applications. 2015.
		\begin{mtheorem}
		    Пусть ребра двудольного графа $G = (L, R, E)$ раскрашены по следующему правилу:
		    \begin{itemize}
		        \item[(*)] для произвольной биклики $C \subseteq G$ и для произвольной пары ребер $(x, y')$ и 
		    $(x', y)$ из $C$ одного цвета $a$, цвет ребра $(x, y)$ тоже $a$.
		    \end{itemize}
		    Зададим распределение вероятностей на ребрах:
		    \begin{itemize}
		        \item $X=[\hbox{левый конец ребра}]$; $Y=[\hbox{правый конец ребра}]$;
		        \item $A=[\hbox{цвет ребра}]$.
		    \end{itemize}
			Тогда $bcc(G) \geq 2^ {\frac{1}{2}(H(A|X) + H(A|Y) - H(A))}$. 
		\end{mtheorem}
	\end{frame}
	
	\section{Выносится на защиту}
	\begin{frame}
		\frametitle{Трудное множество vs оценка Куликова-Юкны}
		\begin{mtheorem}[FS всегда не слабее KJ]
		    В любом графе $G = (V, E)$ среди ребер максимального паросочетания найдется трудное множество размера $\geq \frac{\nu(G)^2}{|E|}$.
		\end{mtheorem}
		
		\begin{mtheorem}[FS иногда значительно сильнее KJ]
		    Для произвольных $\alpha \in [0, \frac{1}{3})$ и $\beta \in (\alpha, \frac{1 + \alpha}{2})$ 
		    при достаточно больших $n$ существует двудольный граф $G = (L, R, E)$ такой, что $|L| = |R| = n$,
		    на котором метод трудного множества дает оценку хотя бы $n^{\beta}$, а оценка Куликова-Юкны 
		    не превосходит $n^{\alpha} + o(1)$. 
		\end{mtheorem}
	\end{frame}
	
	\begin{frame}
		\frametitle{Обобщение информационного метода}
		\begin{mtheorem}
		    Пусть ребра гиперграфа $G = (X_1, X_2, \ldots, X_m, E)$ раскрашены по следующему правилу:
		    \begin{itemize}
		        \item[(*)] для произвольного полного $m$-дольного гиперграфа $C \subseteq G$ и для 
		        произвольного набора ребер $(x_{1,1}, \ldots, x_{1, m}), \ldots, (x_{m,1}, \ldots, x_{m, m})$ 
		        одного цвета $a$, цвет ребра $(x_{1, 1}, x_{2, 2}, \ldots, x_{m,m})$ тоже $a$. 
		    \end{itemize}
		    Зададим распределение вероятностей на ребрах:
		    \begin{itemize}
		        \item $X_i = [i\hbox{-ая вершина ребра}]$;
		        \item $A = [\hbox{цвет ребра}]$.
		    \end{itemize}
		    Тогда выполняется неравенство: $$bcc(G) \geq 2^{\frac{1}{m}(H(A|X_1) + \ldots + H(A|X_m) - (m-1)H(A))}$$
		\end{mtheorem}
	\end{frame}
	
	\begin{frame}
		\frametitle{Используемые теоремы и техники}
		\begin{itemize}
		    \item[1)] При доказательстве первой теоремы была придумана конструкция графа четырехсторонников $\widetilde{G}$.
		    Доказано, что $fool(G) = \omega(\widetilde{G})$, $\max\limits_{K_{r,s}\subseteq G}\{r\cdot s\} = \alpha(\widetilde{G})$ и 
		    $bcc(G) = \chi(\widetilde{G})$. Окончательное утверждение теоремы получается применением 
		    теоремы Турана к графу $\widetilde{G}$.
		    
		    \item[2)] Во второй теореме использовались случайные графы Эрдеша-Реньи. При оценивании 
		    максимального паросочетания применялась лемма Холла, а при оценивании количества ребер -- неравенство Хефдинга.
		    
		    \item[3)] В последней теореме использовалось не Шенноновское информационное неравенство:
		    $$H(A|X_1, B) + H(A|X_2, B) + \ldots + H(A| X_m, B) \leq (m-1)H(A|B)$$
		\end{itemize}
	\end{frame}
	
	\begin{frame}
		\frametitle{Открытые вопросы}
		\begin{itemize}
		    \item[1)] Не удалось доказать, что метод трудных множеств работает \textbf{почти наверное} 
		    (для случайных графов) намного лучше, чем оценка Куликова-Юкны. А именно не получилось доказать
		    $$\sum\limits_{T:T\sim S_0}P\{I_k(T) = 1\ |\ I_k(S_0) = 1\} = o(\mathbb{E}[f_k(G)])$$
		    
		    \item[2)] Не получилось сравнить в общем случае метод трудных множеств с методом информационных неравенств.
		\end{itemize}
	\end{frame}
	
	\section{Конец}
	\begin{frame}
	    \centering \Large{Спасибо за внимание!}
	\end{frame}
	
\end{document}
